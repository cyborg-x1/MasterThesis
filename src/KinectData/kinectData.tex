\chapter{Kinect Analysis}
\graphicspath{{./KinectData/img/}}

As mentioned in the introduction, a closer look to the Kinects data characteristics is taken.


\section{Surface Problems}

There are a few cases when the Kinect is not able to deliver any depth data from a surface.

First of all it does not work with translucent materials like glas or plastics. The projected
points are just fractured away, when they hit in a small angle. It's be possible to look through 
a translucent surface, but only if it's clear and the angle from the camera to the surface is not
to big.
\pic{glas.png}{Glas (Webcam and Pointcloud)}{\label{figure:glas}}{1}{h}

Another problem are mirrors. Mirrors will also fracture the ir spots away, 
so it's not possible to get any depth data (see figure \vref{figure:mirror}).
\pic{Mirror.png}{Mirror (Webcam and Pointcloud)}{\label{figure:mirror}}{1}{h}

If the sun lights up a surface directly it will outshine the ir-spots this will also blank out this surface
in the resulting depth image (see figure \vref{figure:sun}).
\pic{Sun.png}{Sun light on a surface (Pointcloud/Webcam)}{\label{figure:sun}}{0.8}{h}

On surfaces which are to close to the camera, the IR spots will melt together preventing the camera from
identifying their location (see figure \vref{figure:close}) leading to the same result as the previous cases.
\pic{ToClose.png}{Kinect to Close to a Surface (IR sensor and Pointcloud)}{\label{figure:close}}{0.8}{h}


\section{Data Characteristics} 

\subsection{Analyzing Tools}

While this work multiple analyzing tools have been created. Those were created as
nodes to easily connect with the OpenNi driver package in the ROS system.

\subsubsection{DepthImageAnalyzer}
In the beginning a GUI node was created to get in touch with the depth images.
It is able to show a depth image in and if the user clicks into the picture it
will show the corresponding value of the pixel.

%TODO DepthImageAnalyzer Picture

In the later process of this work it was not needed anymore, because showing the point cloud
in rViz showed to be more effective in analyzing data and debugging applications.

\subsection{PointCloud creation with selected topics}
To create point cloud messages for other topics than those from the Kinect node.


\pic{verticalFractals.png}{Vertical Fractals (Pointcloud)}{\label{figure:verticals}}{0.4}{h}


\pic{noise.png}{Noise (Pointcloud)}{\label{figure:noise}}{0.8}{h}

\section{Resolution Depending the Depth}
%Resolution picture


\subsection{}
\pic{availdepths0.png}{Available Depth Values from 0 to 4,8 m}{\label{figure:depths1}}{0.1}{h}
\pic{availdepths1.png}{Available Depth Values from 4,8 m to 9,8m}{\label{figure:depths1}}{0.1}{h}
\pic{LaserDistanceKinectDistance.png}{Distance measured by Laser and by Kinect}{\label{figure:LaserKinect}}{0.9}{h}
\pic{DifferenceForegoing.png}{Distance to the foregoing value in relation to the depth}{\label{figure:LaserKinect}}{1}{h}
 
