\chapter{Kinect Analysis}
\graphicspath{{./KinectData/img/}}

As mentioned in the introduction, a closer look to the Kinects data characteristics is taken
within this document to be able to implement a sign detection method. 
For this work the Kinect RAW fromat was used, each value has 16 bit and shows the
depth for the pixel in millimeters. A depth of zero means that no valid data is available for
this pixel.

\section{Surface Problems}

There are a few cases when the Kinect is not able to deliver any depth data from a surface.

First of all it does not work with translucent materials like glass. The projected
points are just fractured away, when they hit in a small angle. It's be possible to look through 
a translucent surface, but only if it's clear and the angle from the camera to the surface is not
to big.
\pic{glas.png}{Glass (Webcam/Pointcloud)}{\label{figure:glas}}{1}{h}

Another problem are mirrors. Mirrors will also fracture the ir spots away, 
so it's not possible to get any depth data from their surfaces (see figure \vref{figure:mirror}).
\pic{Mirror.png}{Mirror (Webcam/Pointcloud)}{\label{figure:mirror}}{1}{h}

If the sun lights up a surface directly, it will outshine the ir-spots this will also blank out the surface
in the resulting depth image (see figure \vref{figure:sun}).
\pic{Sun.png}{Sunlight on a surface (Pointcloud/Webcam)}{\label{figure:sun}}{0.8}{h}

On surfaces which are to close to the camera, the IR spots will melt together preventing the camera from
identifying their location (see figure \vref{figure:close}) that leads to the same result as in the previous cases.
\pic{ToClose.png}{Kinect to Close to a Surface (IR sensor/Pointcloud)}{\label{figure:close}}{0.8}{h}

\section{Data Characteristics} 

\subsection{Analyzing Tools}

While this work multiple analyzing tools have been created. Those were created as
nodes to easily connect with the OpenNi driver package in the ROS system.

\subsubsection{DepthImageAnalyzer}
In the beginning a GUI node was created to get in touch with the depth images.
It is able to show a depth image in and if the user clicks into the picture it
will show the corresponding value of the pixel.

%TODO DepthImageAnalyzer Picture

In the later process of this work it was not needed anymore, because showing the point cloud
in rViz showed to be more effective in analyzing data and debugging applications.

\subsection{PointCloud creation with selected topics}
Point clouds are very useful for debuging and filtering techniques. They can be displayed in
3D and directly show what happens to the data when a special filter is applied.
To create point cloud messages for other topics than those from the Kinect node it is necessary to
create specific launch files. Launch files are xml-files which are used to start multiple nodes in one
go. 


{%
\tiny{}
\begin{lstlisting}
<!-- Nodelet for creating a pointcloud out of the detector image (for debuging filters) -->
<node pkg="nodelet" type="nodelet" name="PointCloudAdvisor" args="manager" output="screen"/>

<node pkg="nodelet" type="nodelet" name="points_xyzrgb_Advisor" 
	args="load depth_image_proc/point_cloud_xyzrgb PointCloudAdvisor --no-bond">
    <remap from="rgb/image_rect_color"        to="/signDetection/out_rgb" />
    <remap from="rgb/camera_info"             to="/signDetection/camera_info" />
    <remap from="depth_registered/image_rect" to="/signDetection/out_depth" />
    <remap from="depth_registered/points"     to="debug_cloud" />
</node>
\end{lstlisting}
}


As seen in the listing for creating a point cloud from different topics a nodelet manager and a 
depth\_image\_proc/point\_cloud\_xyzrgb nodlet is needed. Inside the node tag of the nodelet
there are the remapping tags, to connect the nodelet with the custom topics.

\subsection{Tool for various data fetching and filter testing}
The tool was named disturbance\_filter\_calculator, the name is still reminding what it was intended to do. It should
filter out the noise by creating a picture of it on a flat surface. This didn't work because of the special autoranging
feature of the Kinect which will be discussed later. Currently the Tool can be used to fetch data points or to
draw a pattern of lines in the RGB image which is mapped over the point cloud. It also can tell the users which data points
are in a given distance and which not by painting the pixels of the RGB image green for being in the specified distance
and red if not. This feature helped to realize about the missing values which are also discussed later. It also includes
some of the first tries to filter and blur the depth image, but as the project advanced it was decided to move the 
filter code to a new clean node in a new package to save compiling time, leaving the disturbance\_filter\_calculator 
node as it was.

%TODO pictures filter_calculator

\section{Disturbances}
While pointing the Kinect to a flat surface the noise fractals (see in figure \vref{figure:noise}) 
seem to reappear at the same position. So it was decided to try to remove them with capturing a image from a flat surface.
and subtract the number of data steps the noise produces. Therefore the kinect is arranged very close to a flat surface 
so that it only will see this surface. Then a picture is created which stores the difference of the data to the 
distance of the surface to be substracted from the following frames.
This was a nice idea but it didn't work as expected. There some special vertical fractals (see in figure \vref{figure:verticals}) 
inside the data which seem to be the result from an autoranging feature of the camera, 
always creating one data step in the picture. The biggest problem is that the size of the noise fractals changes 
with their appearance and their number. The number of those vertical lines ranges from 0 to a spotted maximum of 9. 
They do always appear on a special range, seem to be symetrical or 
nearly symetrical to the center and their number is never even but are not fixed to a special column of pixels. 
\pic{noise.png}{Noise (Pointcloud)}{\label{figure:noise}}{0.8}{h}
\pic{verticalFractals.png}{Vertical Fractals (Pointcloud)}{\label{figure:verticals}}{0.4}{h}

\section{Resolution Depending the Depth} \label{resdepDepth}
When creating the diagram (in figure \vref{figure:LaserKinect}) which shows the occuring values from the camera 
at a specific distance from a plane and parallel surface and the distance measured with a laser distance measurement 
device(see in figure \vref{figure:hilti}),it was realized that many values just do not exist. The camera does never output these values.
After realizing this fact it was decided to look for the available values, their number and the values which never occur.
The figures \vref{figure:depths_1} and \vref{figure:depths_2} show the distance on the left, the available values (black) and 
not availabe values (white) in the middle and the number of the missing values between them on the right. 
Figure \vref{figure:DepthValueDiff} shows the number of missing values in relation to the depth.

\pic{hiltipd38.jpg}{Hilti PD38 Laser Distance Measurement Device}{\label{figure:hilti}}{0.1}{h}
\pic{LaserDistanceKinectDistance.png}{Distance measured by Laser and by Kinect}{\label{figure:LaserKinect}}{0.9}{h}
\pic{availdepths0.png}{Available Depth Values from 0 to 4,8 m}{\label{figure:depths_1}}{0.1}{h}
\pic{availdepths1.png}{Available Depth Values from 4,8 m to 9,8m}{\label{figure:depths_2}}{0.1}{h}
\pic{DifferenceForegoing.png}{Missing values in relation to the depth}{\label{figure:DepthValueDiff}}{1}{h}
