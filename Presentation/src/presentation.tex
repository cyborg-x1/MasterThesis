%%This is a very basic article template.
%%There is just one section and two subsections.
\documentclass{VLKlauck}
\usepackage{graphicx}
\author[]{Christian Holl (24296)}
\institute[HTW Aaalen - Elektronik und Informatik]
{
	Studiengang Computer Controlled Systems\\
	Hochschule Aalen - Technik und Wirtschaft
}

\graphicspath{{./img/}}

\title[]{Kinect Kamera}  

\date[]{\today}

\begin{document}
\maketitle
 
 
\begin{frame}{Kinect}
	\includegraphics[scale=0.7]{KinectSVG.png}
\end{frame} 
\begin{frame}{Ungefiltert (6,5m)}
	\includegraphics[scale=0.4]{0_KinectFarField.png}
\end {frame}
\begin{frame}{Ungefiltert (3,5m)}
	\includegraphics[scale=0.4]{0_KinectNearField.png}
\end {frame}
\begin{frame}{Filter1: Boxed (6,5m)}
	\includegraphics[scale=0.4]{1_BoxedFarField.png}
\end {frame}
\begin{frame}{Filter1: Boxed (3,5m)}
	\includegraphics[scale=0.4]{1_BoxedNearField.png}
\end {frame}
\begin{frame}{Filter1: Median 1 (6,5m)}
	\includegraphics[scale=0.4]{2_BoxedAndFirstMedianFar.png}
\end {frame}
\begin{frame}{Filter1: Median 1 (3,5m)}
	\includegraphics[scale=0.4]{2_BoxedAndFirstMedianNear.png}
\end {frame}
\begin{frame}{Filter1: Problem Blurfilter}
	\includegraphics[scale=0.3]{2_BoxedFirstMedianProblem.png}
\end {frame}
\begin{frame}{Filter1: Zu große Abweichungen von Realpixeln zurücksetzen (6,5m)}
	\includegraphics[scale=0.4]{3_DepthDifferenceFilterBoxedAndFirstMedianFar.png}
\end {frame}
\begin{frame}{Filter1: Zu große Abweichungen von Realpixeln zurücksetzen (3,5m)}
	\includegraphics[scale=0.4]{3_DepthDifferenceFilterBoxedAndFirstMedianNear.png}
\end {frame}
\begin{frame}{Filter1: Problem weg}
	\includegraphics[scale=0.4]{3_DepthDifferenceFilterBoxedAndFirstMedianProblemGone.png}
\end {frame}
\begin{frame}{Filter1: Dezenter Median 2 (6,5m)}
	\includegraphics[scale=0.4]{4_SecondMedianFinalFar.png}
\end {frame}
\begin{frame}{Filter1: Dezenter Median 2 (3,5m)}
	\includegraphics[scale=0.4]{4_SecondMedianFinalNear.png}
\end {frame}
\begin{frame}{Filter1:Problem der Kreiserkennung}
	\includegraphics[scale=0.4]{ProblemKreisErkennung.png}
\end {frame}
\begin{frame}{Filter1:Problem der Kreiserkennung}
	\includegraphics[scale=0.4]{ProblemKreisErkennung1.png}
\end {frame}
\begin{frame}{Filter1:Problem der Kreiserkennung}
	\includegraphics[scale=0.3]{ProblemKreisErkennung2.png}
\end {frame}
\begin{frame}{Filter1:Problem der Kreiserkennung}
	\includegraphics[scale=0.4]{ProblemKreisErkennung5.png}
\end {frame}
\begin{frame}{Filter1:Problem der Kreiserkennung}
	\includegraphics[scale=0.4]{ProblemKreisErkennung4.png}
\end {frame}
\begin{frame}{Filter1:Problem der Kreiserkennung}
	\includegraphics[scale=0.4]{ProblemKreisErkennung3.png}
\end {frame}
\begin{frame}{Filter1:Kreisproblem gefiltertertes Bild}
	\includegraphics[scale=0.4]{ProblemKreisErkennung6.png}
\end {frame}
\begin{frame}{Filter1: Reichweitenfilter}
	Eventuell mit Bildausschnitt in Schildhöhe
	\includegraphics[scale=0.4]{KreisemitReichweitenfilter.png}
\end {frame}
\begin{frame}{Filter1: Größere Werte für Canny 22 61}
	\includegraphics[scale=0.4]{ProblemKreisWeg2261.png}
\end {frame}
\begin{frame}{Filter1: Kreiserkennung\ldots (D\=10cm) (von ca. 40cm - 2m)}
	\includegraphics[scale=0.4]{KreisDone.png}
\end {frame}
\begin{frame}{Forschung: Systematische Fehler in der Kinect}
	\includegraphics[scale=0.2]{FlacheWand.png}
\end {frame}
\begin{frame}{Forschung: Systematische Fehler in der Kinect}
	\includegraphics[scale=0.2]{Kubus.png}
\end {frame}
\begin{frame}{Konzepte für eigene Filter}
	Kreuz-Füllfilter:
		Das Bild wird gleichzeitig von oben nach unten und von rechts nach links
		reihenweise nach Lücken durchsucht.
		Ist eine Lücke gefunden, wird die Länge festgestellt und der Wert des Pixels
		am Ende der Lücke mit dem am Anfang der Lücke verglichen, sind sie gleich 
		oder im Rahmen einer schwelle wird die Lücke gefüllt mit dem medianwert
		gefüllt.
	Stufenglättung:
		Bei einem Sprung beim nächsten Pixel wird der aktuelle Wert festgehalten.
		und auf den nächsten Sprung folgende. Die Pixel zwischen den Werten geglättet
		werden geglättet.
	Ki: Suche nach Fraktalen mittels Wahrscheinlichkeitswerten aus dem Calibrator
\end {frame}
\begin{frame}{Wert Quantisierung bei der Kinect}
	Für den Abstandstest für den Abweichungsgraphen wird über die Pointcloud ein
	Bild gelegt bei dem jeder Pixel der über 5 mm vom eingestellten Wert
	abweicht rot und ein Pixel der sich in der Grenze befindet Grün dargestellt.

	Bei dem Abstand von genau 3 Metern (3000) konnte kein einziger grüner Pixel 
	festgestellt werden, trotz der Schrägstellung der Kinect.
	
	Im Tiefenbild kann auch schon erkannt werden, das eine Quantisierung der Werte
	vorgenommen wird.

	Annahme:	
	Je größer die Abstände zur Kammera werden, desto größer die Quantisierung.
	
\end {frame}

\begin{frame}{Bestätigung der Annahme: Kinect zeigt nur 825 Werte}

	 
	 Überprüft mit einem std::set, welches über ein längeren Zeitraum mit allen
	 Pixelwerten jedes Bildes gefüllt wurde. Die Kamera wurde während dessen im
	 Raum herumbewegt solange bis sich die Anzahl der Werte nicht mehr geändert
	 hat. Hier sieht man anhand der Werte auch deutlich warum Kinect Fusion nur
	 eine Maximaltiefe von 2m für das Tiefenbild berücksichtigt. (siehe nächste
	 Seite)

	 
	
\end {frame}


\begin{frame}{Bestätigung der Annahme: Kinect zeigt nur 825 Werte}
	\includegraphics[scale=0.35]{KinectRectTiefeWertAbstand.png}
\end{frame}

\begin{frame}{Tiefenanalyse des rectified Tiefenbildes}

	\includegraphics[scale=0.5]{KinectRectTiefe.png}

\end {frame}
	 
	 
\begin{frame}{Fehler Kinect2}

	\includegraphics[scale=0.3]{FehlerKinect1.png}

\end {frame}
	 
\begin{frame}{Fehler Kinect3}

	\includegraphics[scale=0.3]{FehlerKinect2.png}

\end {frame}	 


\begin{frame}{Step Map}
	Bei der Step Map handelt es sich um das konvertierte Roh Bild der Kinect, bei
	dem alle die verfügbaren 825 Werte der Kamera in aufeinanderfolgende Zahlen
	konvertiert worden sind. z.B.  317 in 1, 318 in 2, 319 in 3, \ldots 9757 in
	824.
	Dies kann dazu verwendet werden, den Abstand von zwei Pixeln in den
	jeweiligen Ebenen zu bestimmen.
	
	Eventuell taugt dies auch für das Finden vertikaler Artefakte, da ein
	vertikales Artefakt bei eigener Beobachtung immer ein Sprung in einer Ebene
	ist, egal in welcher Tiefe sich das Artefakt befindet.
	
	Ein weiterer Vorteil könnte sich bei der Filterung der anderen Fraktale
	auftreten, welche meist einen ründlichen Blob formen und oft einen Sprung von 1
	aufweisen.
\end{frame}


\begin{frame}{Step Map}
	Ein weiterer Vorteil könnte dadurch entstehen, dass mittels einer bestimmten 
	Calibrierungssoftware die einzelnen Störungen aufgenommen und in der Stepmap
	(unabhängig von der Tiefe) abgezogen werden können.
	
	Fluktuierenden Störungen, könnte man eventuell mittels eines
	Wahrscheinlichkeitsalgorithmuses versuchen auszubessern.
\end{frame}
	 
\begin{frame}{Senkrechte (vertikale) Artefakte}
	
	Es wurde vermutet das zwischen den Pixeln die ab und zu in verschiedener Anzahl
	in den äußeren Bereichen die normalerweise Null sind auftauchen und den
	Vertikalen Fragmenten der Kinect ein Zusammenhang besteht.
	
	Es konnte nicht wirklich nachgewiesen werden das dies der Fall ist.
	Allerdings fällt auf das die äußeren Punkte die nicht Null sind zum Teil mehr
	werden, sofern mehr senkrechte Artefakte da sind. Ob sich jedoch darin eine
	art von Lokalisierungsinfo befindet ist weiterhin unbekannt.
\end {frame}	 
	  

\end{document}
 